\instituicao[a]{Universidade Federal de Santa Catarina} % Opcional
\departamento[a]{Departamento de }
\programa[o]{Programa de } 
\curso{Engenharia de }
\documento[a]{Tese} % [o] para dissertação e trabalho de conclusão de curso [a] para tese
\grau{...} % doutor, mestre, engenheiro, etc.
\titulo{Título}
\subtitulo{Subtítulo (se houver)} % Opcional
\autor{Nome completo do autor}
\local{Florianópolis} % Opcional (Florianópolis é o padrão)
\data{29}{Abril}{2017}
\orientador[Universidade ...]{Prof. ..., PhD}
\coorientador[Universidade ...]{Prof. Dr. ...}
\coordenador[Universidade ...]{Prof. Dr. ...}
\orientadornabanca{sim} % Se faz parte da banca definir como sim
\coorientadornabanca{nao} % Se faz parte da banca definir como sim
\bancaMembroA{Primeiro membro\\Universidade ...}  %Nome do presidente da banca
\bancaMembroB{Segundo membro\\Universidade ...}   % Nome do membro da Banca
\bancaMembroC{Terceiro membro\\Universidade ... \\(Videoconferência)} % Nome do membro da Banca
% \bancaMembroD{Quarto membro\\Universidade ... } % Nome do membro da Banca
% \bancaMembroE{Quinto membro\\Universidade ...}  % Nome do membro da Banca
% \bancaMembroF{Sexto membro\\Universidade ...}   % Nome do membro da Banca
% \bancaMembroG{Sétimo membro\\Universidade ...}  % Nome do membro da Banca

\dedicatoria{Este trabalho é dedicado aos meus colegas de classe e aos meus queridos pais.}

\agradecimento{Inserir os agradecimentos aos colaboradores à execução do trabalho. Inserir os agradecimentos aos colaboradores à execução do trabalho.}

\epigrafe{Texto da Epígrafe. Citação relativa ao tema do trabalho. É opcional. A epígrafe pode também aparecer na abertura de cada seção ou capítulo.}
{(Autor da epígrafe, ano)}

\textoresumo {O texto do resumo deve ser digitado, em um único bloco, sem espaço de parágrafo. O resumo deve ser significativo, composto de uma sequência de frases concisas, afirmativas e não de uma enumeração de tópicos. Não deve conter citações. Deve usar o verbo na voz passiva. Abaixo do resumo, deve-se informar as palavras-chave (palavras ou expressões significativas retiradas do texto) ou, termos retirados de thesaurus da área.}
\palavraschave{Palavra-chave 1. Palavra-chave 2.  Palavra-chave 3. }

\textabstract {Resumo traduzido para outros idiomas, neste caso, inglês. Segue o formato do resumo feito na língua vernácula. As palavras-chave traduzidas, versão em língua estrangeira, são colocadas abaixo do texto precedidas pela expressão ``Keywords'', separadas por ponto.}
\keywords{Keyword 1. Keyword 2. Keyword 3.}
